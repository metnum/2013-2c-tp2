\section{Resultados}
% Deben incluir los resultados de los experimentos, utilizando el formato mas
% adecuado para su presentacion. Deberan especicar claramente a que
% experiencia corresponde cada resultado. No se incluiran aqu corridas de
% maquina. Algo fundamental en su aprendizaje en la materia es la presentacion
% de resultados de forma clara y concisa para el lector

\subsection{Experimento 1: Aumentar el span con carga uniforme}

Para este tipo de experimento lo que hicimos fue fijar todos los parámetros e incrementar el $span$. El propósito de esta prueba era observar como iba evolucionando la fuerza máxima aplicada en los links.\\

Mostraremos dos gráficos en los cuales el set de datos será: $n = 8$, $h = 3$ y $C_i = 5$ para $i \in [1 \dots 7]$ para el primer gráfico y $n = 20$, $h = 3$ y $C_i = 5$ para $i \in [1 \dots 19]$ para el segundo.\\

Nuestra hipótesis basada simplemente en el instinto es que las fuerzas irían aumentando a medida que el $span$ aumenta, ya que la longitud por cada sección aumenta haciendo que los links tengan que soportar mas peso. Veamos lo que realmente sucede:

\begin{center}
\includegraphics[scale=0.8]{archivos/graficos/Fuerza-x-span.png}\\
\end{center}

\begin{center}
\includegraphics[scale=0.8]{archivos/graficos/Fuerza-x-span2.png}\\
\end{center}

Para el primer gráfico la fuerza mayor está en $f_{14}$ hasta que el ancho de la sección es de 5,5 (donde el ratio de ancho vs. altura es de 1,83), luego pasa a ser $f_2$.\\

$f_{14}$ es la viga superior izquierda central (como el puente es simétrico y cualquier asimetría se debe a pequeños errores numéricos, el link $f_{18}$ tiene un valor idéntico).\\

Similarmente se puede realizar un análisis parecido para el segundo gráfico, donde la fuerza mayor ocurre en los links cuya posición es análoga al primer gráfico.\\

Esto indica que, para puentes con secciones más altas que anchas, la vigas superiores centrales se comprimen al soportar todo el peso del puente en una forma desproporcionada.\\

Conforme la sección se hace más ancha, esa carga parece disminuir, hasta que la vigas que soportan más peso son las diagonales extremas, que cargan todo el peso del puente consigo.\\

Y como se puede observar la fuerza máxima disminuye, contrariamente a lo que habíamos formulado en un principio. A la luz de los resultados creemos que esto puede deberse a que a medida que el $span$ crece los angulos en los links diagonales se van achicando haciendo que al multiplicar por las fuerzas, el valor numérico de estas disminuya.\\

Probamos también aumentar el $span$ con carga asimétrica en $n = 20$, poniendo pesos en los links inferiores 4 y 5.Observamos que el comportamiento sigue el mismo que en las figuras anteriores (es decir, una fuerza máxima muy grande que de disminute conforme aumenta el span), y lo único que cambió es que la fuerza máxima es un poco mayor. Nos pareció repetitivo agregar un gráfico tan similar y que dado el contexto no nos aporta mas de lo ya explicado.

\subsubsection{Distribución de la cantidad de fuerzas}

Analicemos ahora la distribución de la cantidad de fuerzas en los casos cuando el $span$ es igual a 200 y a 400. En cada caso el set de datos será: $n = 20$, $h = 3$ y $C_i = 5$ para $i \in [1 \dots 19]$

\begin{center}
\includegraphics[scale=0.8]{archivos/graficos/hist_200.png}\\
\end{center}

\begin{center}
\includegraphics[scale=0.8]{archivos/graficos/hist_400.png}\\
\end{center}

Comparando ambos gráficos se puede ver que la cantidad de fuerzas se va uniformando hacia el valor $0$. Es decir que el promedio de las fuerzas va disminuyendo a medida que el $span$ crece. Esto se condice con el análisis anterior en el cual veíamos como la fuerza mayor disminuía a mayor $span$.

\subsection{Experimento 2: Aumentar el span, pero fijando una carga central}

Ahora veamos que sucede cuando aumentamos el $span$ dejando todo fijo pero colocando solamente un peso significativo en la junta central y fijando las restantes en $0$.

\begin{center}
\includegraphics[scale=0.8]{archivos/graficos/Fuerza-x-span-peso-central.png}\\
\end{center}

Cuando ponemos el peso equivalente a la carga distribuida uniformemente en el vértice central, notamos que el comportamiento es idéntico al experimento anterior en forma, pero la fuerza máxima soporta por un link es considerablemente mayor, y el valor del span desde el cual la fuerza máxima es soportada por las vigas diagonales exteriores es mayor (cuando $span = 600$).\\

Esto se explicaría porque la distribución centralizada del peso fuerza a que las vigas centrales lo carguen sin la ayuda de las vigas inferiores más lejos del centro del puente, que ayudan a soportar el peso cuando el mismo es más uniforme.
