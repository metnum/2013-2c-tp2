\section{Introducción Teórica}

\subsection{Armado de la matriz}

Dada la estructura del puente y que el enunciado del problema pedía averiguar las fuerzas que se aplicaban a cada link, es natural plantear como las incógnitas de nuestros sistemas a las fuerzas. Las ecuaciones serán las juntas sobre las que cada fuerza ejerce su influencia.\\

Veamos que esto forma una matriz cuadrada: La cantidad de links (sobre los cuales las fuerzas se aplican) es $4n - 3$ pero además tenemos las fuerzas que se aplican en los extremos del puente ($v_0$ y $v_1$) y el $h_0$, lo que hace que en total sean $4n$ fuerzas. Luego la cantidad de juntas son $2n$, pero como bien se explicó en el enunciado, para plantear la ecuación de las fuerzas sobre una junta, hace falta descomponer estas en los ejes $x$ e $y$, produciendo que tengamos dos ecuaciones por cada junta, quedándonos $4n$ ecuaciones.\\

\subsection{Numeración de los links}

Para numerar los links empezamos desde abajo hacia arriba y de izquierda a derecha. Los numeramos de esta manera porque nos permitió mantener una relación vecindad numérica con los links que interactúan en las juntas. Y mas importante aún, determina que la matriz sea banda $p$, $q$.\\

De haberlo hecho de otra manera, no podríamos asegurar que la estructura de la matriz sea como el enunciado la describe, ya que la vecindad numérica de la que hablamos en el párrafo anterior se podría perder.

\subsection{Cota para los valores $p$, $q$}

Dada la estructura de la matriz se puede ver que los valores $p$ y $q$ están acotados. Esto está determinado por la cantidad de links que interactúan en las juntas y también por la manera en cómo numeramos los links.\\

Como se verá en la parte de Desarrollo, hay 20 casos posibles de configuraciones de juntas determinadas por los links, cada uno de esos casos determina una ecuación (fila en la matriz). Observando las ecuaciones, tomamos aquella en la cual mas fuerzas se aplican, esta obviamente es la ecuación determinada por la junta central, en cuyo eje $x$ intervienen 5 fuerzas. Estas aparecen en 3 ecuaciones posteriores y en 2 anteriores.\\

Luego en las restantes juntas, en general intervienen entre 3 y 4 fuerzas que a su vez intervienen 3 y 2 ecuaciones posteriores y anteriores respectivamente. Esto nos determina que podemos tomar a 4 como el valor de la cota superior para $p$ y $q$.

\subsection{Demostración de Matriz banda $p$, $q + p$}

En esta sección mostraremos que triangulando la matriz banda $p$, $q$ a lo sumo nos queda una matriz banda $p$, $q + p$\\

Sea una matriz $A \in \mathbb{R}^{n \times n}$ con banda $p$, $q$ y sea el paso $i$ del algoritmo de Gauss de triangulación, tenemos 2 casos:

\begin{enumerate}
    \item {\bf No se hicieron permutaciones anteriores al paso $i$:} Dada la fila-$i$, si quisiéramos permutarla, solo podríamos permutarlas con las filas del $i + 1$ a $i + p$. Ahora si analizamos a la fila-$i$ sabemos que tiene la siguiente forma:

    \begin{displaymath}
        (0, 0, \dots, 0, f_i, f_{i + 1}, \dots, f_{i + q - 1}, f_{i + q}, 0, \dots, 0)
    \end{displaymath}

    O sea, sólo pueden tener valores distintos de $0$ en las posiciones $i$ a $i + q$. El peor caso para permutación sería tomar la fila-${i + p}$. Ahora la fila-${i + p}$ tiene la siguiente forma:

    \begin{displaymath}
        (0, 0, \dots, 0, f_i, f_{i + 1}, \dots, f_{i + q + p - 1}, f_{i + q + p}, 0, \dots, 0)
    \end{displaymath}

    O sea, sólo pueden tener valores distintos de $0$ en las posiciones $i$ a $i + q + p$. Al permutar la fila-$i$ con la fila-${i + p}$ hacemos crecer el valor que la banda $q$ sea igual a $q + p$, el valor de $p$ también no aumenta porque los valores que están a la izquierda de $f_i$ que harían crecer a $p$ son 0.

    \item {\bf Se hicieron permutaciones previamente:} Si ya hubo permutaciones, el caso que nos va a interesar analizar es en el que estamos parados sobre una fila que tiene valores en las posiciones $i$ a $q + p$. Pero haciendo análisis similar al del punto anterior, podemos ver que sólo podemos permutar con las $i + p$ filas siguientes, pero es fácil ver que las posiciones afectadas caen en el rango de $q + p$.
\end{enumerate}

Esbozando un esquema inductivo podemos afirmar que dado una iteración del proceso de triangulación de Gauss sobre la matriz $A$ la submatriz triangular inferior es banda inferior $p$ y la submatriz triangular superior es banda superior $q + p$. Cumpliendo estas normas se puede deducir que la matriz en todo paso de la iteración de Gauss es banda $p$, $q + p$
