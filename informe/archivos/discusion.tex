\section{Discusión y Conclusiones}
% Se incluira aqu un analisis de los resultados obtenidos en la seccion
% anterior (se analizara su validez, coherencia, etc.). Deben analizarse como
% materianimo los lostems pedidos en el enunciado. No es aceptable decir que
% \los resultados fueron los esperados", sin hacer clara referencia a la
% teoremasa a la cual se ajustan. Ademas, se deben mencionar los resul- tados
% interesantes y los casos \patologicos" encontrados.

\subsection{Distribución de los pesos en las juntas}

A lo largo de los distintos experimentos que hicimos, nos llamó la atención que la asignación de los pesos no variaba en demasía el comportamiento general de distribución de fuerzas internas. Es decir, notamos en los experimentos de variación de segmentos del puente, que irrelevante de la forma en que los pesos distribuyan (sea en forma unitaria, con peso concentrado en una sola posición, o distribuido asimétricamente), conforme cambia la forma del segmento de más alto a más ancho, las fuerzas inicialmente son enormes para los juntas superiores horizontales, y luego son las fuerzas diagonales exteriores que cargan todo el peso del puente hasta las bases las más estresadas.\\

Esto es claro verlo en los Experimentos 1 y 2 en donde variábamos la distribución de los pesos y las fuerzas disminuían análogamente en cada gráfico.

\subsection{Cantidad de secciones con $span$ fijo}

Se puede ver que para mayor cantidad de secciones las fuerzas aplicadas sobre cada link van siendo cada vez menos uniformes, haciendo que para algunos links la fuerza sea casi nula y para otros sea excesivamente alta.\\

Creemos que, con segmentos rectangulares cada vez más altos, la capacidad de la junta diagonal de repartir peso uniformemente sobre la estructura se reduce. Intuitivamente se entiende porque en ángulo se parece cada vez más a una junta vertical, y se verifica matemáticamente porque en las ecuaciones de fuerza de los vértices, la fuerza participa menos lateralmente, con lo que las vigas horizontales deben soportar toda la compresión para mantener la estructura rígida.

Algo que al verlo nos pareció natural, pero que aún así no deja de llamarnos la atención, fue que en esta serie de gráficos de distribución de las fuerzas, la repartición de los links que estaban tensionados contra los que estaban en compresión fue casi simétrica. La diferencia, creemos, debe ser por errores numéricos.

\subsection{Los links centrales son los que soportan mas fuerzas}

Tal vez nos faltó agregar una serie de gráficos para hacer mas explícito este punto, pero podemos asegurar que en la mayoría de los casos observados pudimos comprobar que esto se cumplía.
